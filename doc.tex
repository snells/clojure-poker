\title{Ohjelmointikielet ja -paradigmat 2017}
\author{
  Sakari Sn\"all \\
  Antti Vainikka
\\
}
\date{\today}

\documentclass[12pt]{article}

\usepackage[utf8]{inputenc}
\begin{document}
\maketitle

\section{Johdanto}
Toteutimme harjoitustyön kolmannen tehtävän.
Valitsimme toteutettavaksi peliksi pokerin koska järkevän toteutuksen tekeminen tutustuttaisi meidät kielen tapaan luoda struktuureita,
lukea syötettä standardista syöteputkesta ja joutuisimme toteuttamaan pelijatkumon funktionaalisesti. 

\section{Funktionaalinen paradigma}
Funktionaalinen paradigma on ohjelmointiparadigma. Funktionaalisen paradigman omainaisuuksia ovat esimerkiksi muuttumattomat tietorakenteet ja laiskalaskenta.
On vaikea määrittää onko ohjelmoinkieli funktionaalinen vai ei. Useimmissa ohjelmointikielissä on olemassa mahdollisuus luoda anonyymejafunktioita ja sen pohjalta on vedetty johtopäätös, että kieli olisi funktionaalinen. Esimerkiksi Java8 kutsutaan funktionaaliseksi koska siihen lisättiin lyhenne anonyymille funktiolle. Todellisuudessa Java8 luodaan objektin instanssi jossa 

%pseudokoodi esimerkki funktionaalisesta loopista

%pseudokoodi esimerkki imperatiivisesta loopista


\subsection{Mahdolliset hyödyt}
\subsection{Mahdolliset vaikeudet}
\section{Clojure}
Clojure on ohjelmointikieli, joka on saanut paljon vaikutteita Common Lispistä. 
\subsection{Funktionaalinen Clojure}

\section{Harjoitustyö}

\subsection{Funktionaalisuudesta saadut hyödyt}

\subsection{Funktionaalisuuden haitat}

\bibliographystyle{abbrv}
\bibliography{main}


\end{document}

